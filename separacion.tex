\documentclass[11pt,openany]{book}
%Paquete de idioma (Español)
\usepackage[spanish]{babel}
%utf8 (reconoce los acentos españoles)
\usepackage[utf8]{inputenc}
%Paquete de fuentes (?)
\usepackage{amsfonts}
%Paquete de gráficos
\usepackage{graphicx}
%No se para que es
\usepackage{tipa}
%Distribución de la página (margen, pié y encabezado)
\usepackage[left=2cm, right=2cm, top=2.00cm, bottom=2.00cm]{geometry}

\usepackage{enumerate}
\usepackage{enumitem}
\usepackage{listings}
\usepackage{amsmath}
\usepackage{physics}
\usepackage{amsmath,amssymb}
\usepackage{mathtools}
\setlength{\parindent}{12pt}
\usepackage[makeroom]{cancel}

	
\begin{document}
	\chapter{Separación de variables genérico}

\noindent Supongamos que estamos en $\mathbb{R}^2$, y nos piden resolver mediante separación de variables de la siguiente forma:
\\[8pt]

\textsl{Ejercicio: Encuentra las soluciones de} 
	\begin{equation}
	 \Delta u=0
	\end{equation} 
	
	\textsl{usando método de separación de variables. Clasifica las soluciones en función del signo de autovalor $\lambda$.}
\\[8pt]

\noindent Lo primero que tenemos que hacer es reescribir la ecuación teniendo en cuenta que u depende de dos variables.
\begin{equation}
	\Delta u=0 \longleftrightarrow \pdv{u^2}{x^2}+\pdv{u^2}{y^2}=0
\end{equation}
\noindent Ahora separamos u en dos variables, que serán x e y.
\begin{equation}
u(x,y)=v(x)w(y)
\end{equation}

\noindent A esta nueva representación le aplicamos la ecuación que nos piden, en (1.1), y llegamos a lo siguiente:
\begin{equation} \nonumber
\begin{cases}
\pdv{u^2}{x^2}=v''(x)w(y) \\ \pdv{u^2}{y^2}=v(x)w''(y)
\end{cases}
\longleftrightarrow \ \pdv{u^2}{x^2}+\dv{u^2}{y^2}=0 \longleftrightarrow \ v''(x)w(y)+v(x)w''(y)=0 \quad \frac{V''(x)}{V(x)} = \frac{-w''(y)}{w(y)} = \lambda 
\end{equation}
Ahora dependiendo del valor de $\lambda$, tendremos unos casos u otros.

\begin{itemize}
	\item Si $\lambda$=0:
	\begin{equation} \nonumber
	v(x)=Ax + B
	\end{equation}
	\begin{equation} \nonumber
	w(y)=Cy + D
	\end{equation}
	\item Si $\lambda>0$:
	\begin{equation} \nonumber
	w(y)=A\cos(\sqrt{\lambda}y)+B\sin(\sqrt{\lambda}y)
	\end{equation}
	\begin{equation}  \nonumber
	v(x)=Ce^{(\sqrt{\lambda}x)}+De^{-(\sqrt{\lambda}x)}
	\end{equation}
	\item Si $\lambda<0$, se cambian los papeles del apartado anterior, y teniendo en cuenta que ahora dentro de la raiz tenemos $-\lambda$.
\end{itemize}
Para llegar a esto se han usado métodos de resolución de EDOs. Una vez llegado aquí, se aplican a estas soluciones las condiciones que tengamos y nos quedará siempre una única opción.
\chapter{Algunos ejercicios resueltos por mi}
\textsl{Sea $\Omega$ = $\{(x,y)\in \mathbb{R}^2: 0<x<a, 0<y<b\}$ resolver el problema con condiciones de contorno mixtas:}
\begin{equation} \nonumber
\begin{cases}
\Delta u=0 & \text{en $\Omega$}\\ u(x,0)=0, u(x,b)=g(x) & \text{para } 0\leq x\leq a \\ u(0,y)=u_x(a,y)=0 & \text{para } 0\leq y\leq b
\end{cases}
\end{equation}
\textsl{donde g $\in \mathcal{C}^2(\mathbb{R})$, g(0)=0, g'(a)=0, ($u_x= \pdv{u}{x}$)}
\\[11pt]
\noindent Empezamos haciendo el estudio de casos de valores de $\lambda$, y nos saldrá que el único caso posible es el de $\lambda<0$.
En la parte espacial, tendríamos
\begin{gather*} 
V(x)=A\cos(\sqrt{|\lambda|}x)+B\sin(\sqrt{|\lambda|}x) \quad V(0)=0 \longleftrightarrow A = 0\\
V'(x)= \sqrt{|\lambda|}B\cos(|\lambda|x) \xrightarrow{x=a} \sqrt{|\lambda|}B\cos(|\lambda|a) = 0 \longleftrightarrow \sqrt{|\lambda|} a = (n+ \frac{1}{2}) \pi \longleftrightarrow \sqrt{|\lambda|} = \frac{(n+ \frac{1}{2}) \pi}{a} \\
V(x)=A\sin(\frac{(n+ \frac{1}{2}) \pi}{a}x)
\end{gather*}
Ahora, para la parte que corresponde a la Y
\begin{gather*}
W(y)=Ce^{(\sqrt{|\lambda|}y)}+De^{-(\sqrt{|\lambda|}y)} \xrightarrow{y=0,w=0} C+D=0 \\
W(y)= Ce^{(\frac{(n+ \frac{1}{2}) \pi}{a}y)}-Ce^{-(\frac{(n+ \frac{1}{2}) \pi}{a}y)}=C(e^{(\frac{(n+ \frac{1}{2}) \pi}{a}y)}-e^{-(\frac{(n+ \frac{1}{2}) \pi}{a}y)})\\
\text{Esto, por definición de seno hiperbólico, nos queda } W(y)=2C\sinh(\frac{(n+\frac{1}{2})\pi}{a})
\end{gather*}
Ahora, para formar nuestra solución definitiva, multiplicamos V y W, pero con la particularidad de que por comodidad el producto de las constantes (\textsl{A por parte de V, 2C por parte de W}) pasará a ser denotado por A, y no perderemos nada en este paso pues siguen siendo constantes.
\\[11pt]
Ahora, construimos nuestra solución final
\begin{gather*}
u(x,y)= \sum\limits^\infty_{n=1} A_n\sin(\frac{(n+ \frac{1}{2}) \pi}{a}x)\sinh(\frac{(n+\frac{1}{2})\pi}{a}y) \text{ Si ahora le aplicamos la condición u(x,b)=g(x) }\\
u(x,b)= \sum\limits^\infty_{n=1} A_n\sin(\frac{(n+ \frac{1}{2})=g(x) \pi}{a}x)\sinh(\frac{(n+\frac{1}{2})\pi}{a}b)\\
\text{ Donde la parte que depende del sinh ahora es una constante y la agrupamos con $A_n$}\\
u(x,b)=\sum\limits^\infty_{n=1} A_n\sin(\frac{(n+ \frac{1}{2}) \pi}{a}x)=g(x)\\
\text{Por el enunciado sabemos que en x=0 la funcion vale 0, y que si derivamos y evaluamos en x=a, también vale 0}\\
u(0,b)=\sum\limits^\infty_{n=1} A_n\sin(\frac{(n+ \frac{1}{2}) \pi}{a}0)=g(0)=0 \text{ (Obvio se cumple pues el seno anula todo)}\\
u'(x,b)=A_n\frac{(n+ \frac{1}{2}) \pi}{a}\cos(\frac{(n+ \frac{1}{2}) \pi}{a}x) \\
u'(a,b)=A_n\frac{(n+ \frac{1}{2}) \pi}{a}\cos(\frac{(n+ \frac{1}{2}) \pi}{\cancel{a}}\cancel{a})=g'(a)=0\\
\text{Que también es cierto puesto que ese coseno se anula siempre, luego solo nos falta calcular $A_n$}\\
A_n=\int_{0}^{a} g(x)\sin(\frac{(n+ \frac{1}{2}) \pi}{a}x)\\
\text{Y aquí acaba el ejercicio puesto que no tenemos ninguna g(x) específica}
\end{gather*}
\end{document}
